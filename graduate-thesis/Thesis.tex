%% ----------------------------------------------------------------
%% Thesis.tex -- MAIN FILE (the one that you compile with LaTeX)
%% ----------------------------------------------------------------

% Set up the document
\documentclass[a4paper, 11pt, oneside]{Thesis}  % Use the "Thesis" style, based on the ECS Thesis style by Steve Gunn
\graphicspath{Figures/}  % Location of the graphics files (set up for graphics to be in PDF format)
\usepackage[table]{xcolor}
% Include any extra LaTeX packages required
\usepackage[square, comma, sort&compress]{natbib}  % Use the "Natbib" style for the references in the Bibliography
\usepackage{verbatim}  % Needed for the "comment" environment to make LaTeX comments
\usepackage{vector}  % Allows "\bvec{}" and "\buvec{}" for "blackboard" style bold vectors in maths
\usepackage{hyperref}
\hypersetup{urlcolor=blue, colorlinks=true}  % Colours hyperlinks in blue, but this can be distracting if there are many links.
\usepackage{tgcursor}
\usepackage{romannum}
\usepackage{pdfpages}
\usepackage{fancyhdr}
\usepackage{textcomp}

% \usepackage{subcaption}
%% ----------------------------------------------------------------
\begin{document}
\frontmatter      % Begin Roman style (i, ii, iii, iv...) page numbering

% Set up the Title Page
\title  {An evaluation of touch-based music sequencer apps on iPad}
\authors  {\texorpdfstring
            {{Ke Ding}}
            {Ke Ding}
            }
\addresses  {\groupname\\\deptname\\\univname}  % Do not change this here, instead these must be set in the "Thesis.cls" file, please look through it instead
\date       {\today}
% \subject    {}
\keywords   {}

\maketitle
%% ----------------------------------------------------------------

\setstretch{1.3}  % It is better to have smaller font and larger line spacing than the other way round

% Define the page headers using the FancyHdr package and set up for one-sided printing
\fancyhead{}  % Clears all page headers and footers
\rhead{\thepage}  % Sets the right side header to show the page number
\lhead{}  % Clears the left side page header

\pagestyle{fancy}  % Finally, use the "fancy" page style to implement the FancyHdr headers

%% ----------------------------------------------------------------
% Declaration Page required for the Thesis, your institution may give you a different text to place here
\Declaration{

\addtocontents{toc}{\vspace{1em}}  % Add a gap in the Contents, for aesthetics

I, Ke Ding, declare that this thesis titled, `An evaluation of touch-based music sequencer apps on iPad' and the work presented in it are my own. I confirm that:

\begin{itemize}
\item[\tiny{$\blacksquare$}] This work was done wholly or mainly while in candidature for a research degree at this University.

\item[\tiny{$\blacksquare$}] Where any part of this thesis has previously been submitted for a degree or any other qualification at this University or any other institution, this has been clearly stated.

\item[\tiny{$\blacksquare$}] Where I have consulted the published work of others, this is always clearly attributed.

\item[\tiny{$\blacksquare$}] Where I have quoted from the work of others, the source is always given. With the exception of such quotations, this thesis is entirely my own work.

\item[\tiny{$\blacksquare$}] I have acknowledged all main sources of help.

\item[\tiny{$\blacksquare$}] Where the thesis is based on work done by myself jointly with others, I have made clear exactly what was done by others and what I have contributed myself.
\\
\end{itemize}


Signed:\\
\rule[1em]{25em}{0.5pt}  % This prints a line for the signature

Date:\\
\rule[1em]{25em}{0.5pt}  % This prints a line to write the date
}
\clearpage  % Declaration ended, now start a new page

%% ----------------------------------------------------------------
% The "Funny Quote Page"
\pagestyle{empty}  % No headers or footers for the following pages

\null\vfill
% Now comes the "Funny Quote", written in italics
\textit{``We tend to overestimate the effect of a technology in the short run and underestimate the effect in the long run.''}

\begin{flushright}
Roy Amara, leader at the Institute for the Future
\end{flushright}

\vfill\vfill\vfill\vfill\vfill\vfill\null
\clearpage  % Funny Quote page ended, start a new page
%% ----------------------------------------------------------------

% The Abstract Page
\addtotoc{Abstract}  % Add the "Abstract" page entry to the Contents
\abstract{
\addtocontents{toc}{\vspace{1em}}  % Add a gap in the Contents, for aesthetics

With the repid development of technoldge, mobile devices have become the new playground for musicians to express themselves. With a variety of sensors, as well as the exponentional growth in the processing power, iPad offer an attracitve platform for music performing. Thousands of music applications have been developed for the iPad. Music sequencer applications, as one of the major category of music making applications, have seen a lot of derivation and innovation. But the question is among those novel music sequencer interface which one supports musicians performance and stimulates their creativities? This thesis is trying to answer the above question from the perspective of musicians. The research was seperated into two consecutive study. In the first study, we investigated 55 music sequencer applications from App Store and created an interface taxonomy, in which music sequencer applications were divided into three groups according to the mapping of pitch, trigger and timber. In the second study, three most representative applications from each group were selected and tested by musicians to evaluate the pros and cons in the different design approaches. By employing the MPX-Q questionnaire, we quantitatively analyzed the strength and weakness of each sequencer applications. Follow by, a qualitative study was conducted to reveal the reason behind those design.

}

\clearpage  % Abstract ended, start a new page
%% ----------------------------------------------------------------

\setstretch{1.3}  % Reset the line-spacing to 1.3 for body text (if it has changed)

% The Acknowledgements page, for thanking everyone
\acknowledgements{
\addtocontents{toc}{\vspace{1em}}  % Add a gap in the Contents, for aesthetics

It has been a very interesting and challenging semester. I have many to thank for the support and guidence.

I would like to thank my supervisor Dr.Ben Swift for his   patience and dedication throughout the project. My appreciation goes to Ben for his encouragement and selfless support. He always guided me when I was confused and pushed me at the time I needed most. I would like to thank Pro. Weifa Liang, as course convenor, he provided many useful guidence to the presentation and structure of academic report.

My appreciation goes to the participants who spent their time in the user study and gave me many valuable suggestions. Special thanks to Kirean Brown for his passion and experience. Jiang, for his suggestion on managing papers and academic report writing. Rich, for his advertisment to recuit participants on facebook.

Finally, I would like to thank my family, friends and classmates who have create a comfort and cozy study environment. In particular, I would like to thank Eva for her accompany and support thoughout the project. Thank you all.
}
\clearpage  % End of the Acknowledgements
%% ----------------------------------------------------------------

\pagestyle{fancy}  %The page style headers have been "empty" all this time, now use the "fancy" headers as defined before to bring them back


%% ----------------------------------------------------------------
\lhead{\emph{Contents}}  % Set the left side page header to "Contents"
\tableofcontents  % Write out the Table of Contents

%% ----------------------------------------------------------------
\lhead{\emph{List of Figures}}  % Set the left side page header to "List if Figures"
\listoffigures  % Write out the List of Figures

%% ----------------------------------------------------------------
\lhead{\emph{List of Tables}}  % Set the left side page header to "List of Tables"
\listoftables  % Write out the List of Tables

%% ----------------------------------------------------------------
\setstretch{1.5}  % Set the line spacing to 1.5, this makes the following tables easier to read
\clearpage  % Start a new page
\lhead{\emph{Abbreviations}}  % Set the left side page header to "Abbreviations"
\listofsymbols{ll}  % Include a list of Abbreviations (a table of two columns)
{
% \textbf{Acronym} & \textbf{W}hat (it) \textbf{S}tands \textbf{F}or \\
\textbf{NIME} & \textbf{N}ew \textbf{M}usical \textbf{I}nstrument \textbf{E}xpression \\

\textbf{MPX-Q} & \textbf{M}usicians's \textbf{P}erception of the E\textbf{x}periential \textbf{Q}uality\\

\textbf{DMI} & \textbf{D}igtal \textbf{M}usical \textbf{I}nstrument \\
}

%% ----------------------------------------------------------------
% \clearpage  % Start a new page
% \lhead{\emph{Physical Constants}}  % Set the left side page header to "Physical Constants"
% \listofconstants{lrcl}  % Include a list of Physical Constants (a four column table)
% {
% % Constant Name & Symbol & = & Constant Value (with units) \\
% Speed of Light & $c$ & $=$ & $2.997\ 924\ 58\times10^{8}\ \mbox{ms}^{-\mbox{s}}$ (exact)\\
%
% }

%% ----------------------------------------------------------------
% \clearpage  %Start a new page
% \lhead{\emph{Symbols}}  % Set the left side page header to "Symbols"
% \listofnomenclature{lll}  % Include a list of Symbols (a three column table)
% {
% % symbol & name & unit \\
% $a$ & distance & m \\
% $P$ & power & W (Js$^{-1}$) \\
% & & \\ % Gap to separate the Roman symbols from the Greek
% $\omega$ & angular frequency & rads$^{-1}$ \\
% }
% % %% ----------------------------------------------------------------
% % End of the pre-able, contents and lists of things
% % Begin the Dedication page
%
% \setstretch{1.3}  % Return the line spacing back to 1.3
%
% \pagestyle{empty}  % Page style needs to be empty for this page
% \dedicatory{Dedicated to my parents}
%
% \addtocontents{toc}{\vspace{2em}}  % Add a gap in the Contents, for aesthetics
%
%
% %% ----------------------------------------------------------------
\mainmatter	  % Begin normal, numeric (1,2,3...) page numbering
\pagestyle{fancy}  % Return the page headers back to the "fancy" style

% Include the chapters of the thesis, as separate files
% Just uncomment the lines as you write the chapters

\pagestyle{fancy}
\rhead{\thepage}
\lhead{Introduction}
\chapter{Introduction}
% Add some introduction on sequencer
Music sequencer, as an instrument, has been studied by musicians for a long time. In section \ref{subsec: history} the development of sequencer was introduced.

With the rapid development of digital audio technology, computers are playing an increasingly important role in music, and providing unprecedented opportunities for people to create and manipulate sound. The iPad, a portable computer with touch screen, provides a new platform for musicians to express themselves (see section \ref{subsec: iPad}).

Since thousands of new musicial interfaces built on computers have been created and released to the world, it is natural to ask, what kinds of musical interfaces are taking better advantage of computers? To answer this question, a new community called NIME emergeed from the establised field of human-computer interaction (HCI) \citep{Reference16}. And in NIME, the evaluation of musical interface is mainly focused on expressivity, control and aethetic three aspects \citep{Reference0}. 

% Talk anout creativity, control and aethetics....
\section{Background}
\label{sec: backgound}

\subsection{Development of Music Sequencer}
\label{subsec: history}
By modern definition, a music sequencer (or sequencer) is a hardware device or software application which can handle music materials such as notes, sounds and many other forms of performance information. It is widely used in performing electronic music, and used as a processing tool in music composing. The history of music sequencer can refer to 9th century, the earlier sequencers worked as sound producing devices and depended on the preset inputs. A famous example of early stage music sequencer is player piano, a self-playing piano which play the music pre-recorded on a piano roll. When the technology matured, more forms of sequencer emerged. Analog sequencer which took advantage of analog eletronics was the first sequencer designed to live performance except music composition. Step sequencer, also known as drum machine, distriputed the note into steps with equal time-interval, which free musicians from acurately timing of each note.

In morden times, digital sequencer implemented on computers become popular. The advent of Musical Instrument Digital Interface (MIDI) made the digital sequencer with great vitality. In recent years, the popularity of mobile devices brought music sequencers to a new platform and many new elements have been put into practice.

\subsection{iPad: a new playground for musicians}
\label{subsec: iPad}
% we will first brefly introduce the ipad, and then illustrate the strength of ipad. also the reason why we major investigate on ipad

The iPad, a tablet computer with touchscreen display, has quickly occupied the market all around world since it's first release in 2010 \citep{Reference2}. The emergence of iPad have provided a new platform for users to explore digital world \citep{Reference1}. After 7 generations, the usage of iPad has shifted from the extension of iPhone to a powerful pruductivity tool. In this shift, thousands of applications which was designed to utilise the larger touch screen has emerged. According to \citeauthor{lifewire}, there are over 1.5 million apps are currently hosted in the App Store and more than half of those apps are specifically designed for iPad \citep{lifewire}.

Since the first release of iPad, there are practices to utilise the large tangible screen and wide variety of sensors of this cross-time product. \citeauthor{Reference8.4} designed \textit{Magic Fiddle}, a new musical instrument, which combined the physical gesture of users and graphical display of iPad together. \citeauthor{Reference19} explored the possibility of using iPad as a percussive instrument and used iPad's network feature to ecourage cohesive improvisation \citep{Reference19}.

\section{Related Work}

A lot of work have been down on evluating the interaction between users and mobile devices such as iPhone. However, there haven't been a paper specificly analyze musical instrument implemented on iPad. \citeauthor{Reference21} evaluated the live music-making on computer through discourse analysis and turing test \citep{Reference21}. A questionnaire-based evaluation method was proposed to evaluate the musical instruments, especially the new forms of instruments from NIME \citep{Reference0}.

Unexpectedly, music sequencers as the top three most popular instruments in iOS musical applications \citep{Reference14}, has not attracted much attention. We can barely find papers related to recent years development of music sequencers application. The most related work was \textit{Block Jam} (see figure \ref{fig: Block Jam}), a sequencer with tangible interface consisited of several physical blocks \citep{Reference20}.

\bigskip
\begin{figure}[h]
  \includegraphics[width=12 cm]{images/blockjam.jpg}
  \centering
  \caption{Block Jam: music sequencer consist of a cluster of blocks}
  \label{fig: Block Jam}
\end{figure}
\bigskip

\section{Research motivation and goals}

While musical interfaces have been studied for a long time, there have emerged thousands of novel twists on \textquotedblleft{grid-based}\textquotedblright music sequencer. And to our knowledge there is currently no paper investigate the situation of this certain kind of musical application on App Store. What's more, there is no consensus on using what's method to evaluate those newborn musical application on mobile devices. This work is a first attempt to classify the music sequencers on iPad and adopt the evaluation method(MPX-Q Questionnaire) designed for NIME community. The goal of this thesis is to evaluate touch-based music sequencer apps on iPad by:
\begin{flushleft}
$\bullet$ Creating an interface taxonomy of current music sequencer apps on the iOS app store.\\
$\bullet$ Performing a HCI user study to measure user experience and musicians performance with different interface design approaches.\\
$\bullet$ Proposing design guidlines for musicians, developers and researcher for creating musical interface in the future.
\end{flushleft}

\section{Thesis structure}

In the next chapter, literature relevant to the research topic was introduced, so as to establish a theoretical framework of the research (see Chapter \ref{ch: chapter 2}). And the research project was divided into two consecutive studies. In the first study, we analyzed the music sequencer applications (designed for iPad) on App Store and create an interface taxonomy (see Chapter \ref{ch: chapter 3}). Then base on the classification of music sequencer interfaces, we selected one most representative application from each category and conducted an user study to evlauate the effect of different interface design (see Chapter \ref{ch: chapter 4}). In Chapter \ref{ch: chapter 5}, we discussed the results and provided a conclusion of the study as well as the future work.
 % Introduction

\pagestyle{fancy}
\rhead{\thepage}
\lhead{Literature Review}
\chapter{Literature Review}
\label{ch: chapter 2}

\section{Mobile Music}

With the increasing popularity of mobile device such as smart phone and tablet, a new research field called Mobile Music emerged \citep{Reference4}. According to the definition by \citeauthor{Reference6}, \textit{Mobile Music} wich employing portable technolodge does not only include the scope of playing music, but also involve music composing, synthesizing and sharing\citep{Reference6}.

In the last sixteen years, there is a growing number of researchers start concerning the development of applications in mobile devices. This new trend was first highlighted by \citeauthor{Reference12} after analysing 98 NIME procedding papers related to mobile music during the period from 2002 to 2012\citep{Reference12}.

The expanding capabilities of mobile devices inspired researchers to exploit the new features.The wireless network ability of mobile device is the first area attract researchers' attention. TunA is the first practice of building connection among PDA users though wireless network\citep{Reference7}. By accessing the playlists of nearby users, TunA help users in same network to exchange their music. \citeauthor{Reference5} extended \citeauthor{Reference7}'s work from music sharing towards collaborative musical creation \citep{Reference5}. \citeauthor{Reference5} propsed a system which exploits ad-hoc wireless networks to allow a community of people using their PDA to work on the same piece of music \citep{Reference5}. Some research started from a different approach by investigating the possibility of utilizing the touch screen on the mobile devices. Geiger designed a paradigm for using touch screen on mobile device like iPaq \citep{Reference9, Reference10}.
MoGMI, which stand for Mobile Gesture Music Instrument, is a research project focused on using the accelorometer inside the mobile phone to perform music. Through examing three different axis mapping models, \citeauthor{Reference11} explored how to turn mobile phone into a standard instrument. Smule Ocarina is the most successful mobile musical artifact, which take advantanges of the global popularity of iPhone \citep{Reference8.1}. It leveraged the microphone to take input from breath, and combined with command from the multitouch screen to mimic the physical interaction of ocarina. Besides, Smule Ocarina also utilizing the GPS module to connect users all around the world and create a new social experience \citep{Reference8}.


\section{Musical Interaction Patterns}
% in this section we are going to introduce several major interaction patternn in mobile devices bath on \cite{Reference4}
Musical interaction patterns, also konw as design patterns, are common solutions for developers to design a specific interface,like music sequencer. \citeauthor{Reference4} stated since designer can reuse the proven discipline in their work, design patterns can assisit multidisciplinary design, improve communication between designers and facilaitae knowledage transfer between teams with different background \citep{Reference4}. In \citeauthor{Reference4}'s work, following four most common music interaction patters on mobile devices were given:
1). Natural Interaction. 2). Event Sequencing. 3). Process Control 4). Sound Mixing.
In which, event sequencing was the second most popular interaction patterns. The general decription of event sequencing pattern was illustrated as: by editing the sequence of musical event which maybe individual notes, several piece of samples or parameters that can modify the sound of music \citep{Reference4}. In \citeauthor{Reference13}'s paper, sequencer was put into an independent category of musical application on App Store, and it's nature of mapping was briefly discussed.


\section{Evaluation of digital musical instruments}

Digital musical instruments (DMIs) refer to instruments whose sound are generated digitally. It is not uncommon to ask what does evaluation means in the context of digital musical instruments. But as \citeauthor{Reference25} mentioned, evluating the expressivness and creativity of an musical interface were very difficult \citep{Reference25}. \citeauthor{Reference25}'s paper followed by providing a methodology based on discourse analysis. An evaluation framework was given by \citeauthor{Reference22}, in which DMIs were evaluated from four interdependent prosepective: audience, performer, designer and manufacturer \citep{Reference22}. Also, three general design goals were listed at \citeauthor{Reference22}'s paper, which were \textit{Enjoyment, Playability and Robustness}. \citeauthor{Reference23} proposed a process to evaluate DMIs from a performer's view \citep{Reference23}. A case study conducted by \citeauthor{Reference24} was focued on the expressiveness and mapping of DMIs. Recently, by reviewed 89 papers published in NIME from 2012 to 2014, \citeauthor{Reference26} pushed forward the discussion to how to better use the evaluation tools to improve the design of DMIs \citep{Reference26}.

\clearpage
 % Background Theory

\chapter{Study 1: Classification of music sequencer}
\label{ch: chapter 3}

A big scale study was conducted to create an interface taxonomy of current music sequencer apps on the iOS App Store. In total, 55 music sequencer applications on App Store have been examined (see Appendix \ref{app:Appendix A}). Several search criteria are implemented to locate music sequencer on the App Store (see Section\ref{subsec: search criteria}). After analyzing those music sequencer apps, we proposed classification criteria based on the design of the user interface (see Section \ref{sec: classify criteria}). The 55 music sequencer applications were classify into 3 major groups according to the classification criteria (see Section\ref{sec: result}).

\section{Method}
\label{sec:method}

In total, 71 musical iOS applications associated with music sequencer had been downloaded from App Store. After examined and discussed with my supervisor Ben Swift, 16 applications were removed from the study list either because the application can hardly be classified as music sequencer or because the application was not designed for iPad. The rest 55 music sequencer applications were studied in detailed.

\subsection{Search Criteria}
\label{subsec: search criteria}

Base on \citeauthor{Reference13}'s study which created a whitelisted words for music sequencer, keywords such as \textit{Sequence, Sequencer, Groovebox, Beatbox, Step, MIDI, Pattern, Tempo, BPM, Machine} were used to search on the App Store. Before each application been downloaded, it's description had been briefly overviewed to make sure it was designed for music purpose. Also, in the searching criteria, \textquotedblleft{iPad only}\textquotedblright was chose and results were sorted under the relevance of keywords.

\subsection{Classification criteria}
\label{sec: classify criteria}
The different approaches of interacting with the applications were used to classify the user interface of the music sequencer applications into several categories. The mappings of the sequencer were broke down into 4 operations,
which were \textit{changing pitch, triggering sound, timing and changing timber}.

\textbf{Changing Pitch.} Becasue the way most traditional instruments' pitch were changed discretely, for example, piano, guitar and violin. The majority of musical application including sequencer follow this trend. Besides, pitch is dominated by grid-like, buttom-to-top mapping in music sequencer hardware. Therefore, grid-based, buttome-to-top and discrete pitches layout is widely adopted.

\bigskip
\begin{figure}[h]
  \includegraphics[width=12 cm]{images/Beatwave.PNG}
  \centering
  \caption{Beatwave: grid-based, buttome-to-top and discrete pitch layout}
  \label{fig: Beatwave}
\end{figure}
\bigskip

 Figure \ref{fig: Beatwave} is a good example of this classic interface, in which the interface is divided into 16x16 grids. The time, which is separated into 16 steps, only moves one step at a time from left to right. The blue vertical line works as a reminder of current time, and also indicates what is coming next(in the next step). The white square, on the other hand, represents the sound of a certain instrument. In this case, it represents an electrical sound called \textit{FUTURE}. The column in each step is divided into 16 scales and which are the pitches of the instrument. The white squares located in the top of the grids are high pitch sound of the instrument, on the contrary, the pitch of the sound from the bottom is relatively low.

 \bigskip
 \begin{figure}[h]
   \includegraphics[width=12 cm]{images/SoundZen.PNG}
   \centering
   \caption{SoundZen: grid-based, right-to-left and discrete pitch layout}
   \label{fig: SoundZen}
 \end{figure}
 \bigskip

 However, not all the grid-based sequencer applications increase pitch from bottom to top. There is a small portion of sequencer increase pitch from left to right. For instance, SoundZenHD used a left-to-right pitch mapping (see figure: \ref{fig: SoundZen}).

In addition to the discrete pitch mapping, there are attempts to implement the continuous pitch. \textit{CSketch Lite} followed the classic grid-based layout, but it implements continuous sequencing (see Figure \ref{fig: CSketch}). By implementing the continuous sequenceing, \textit{CSketch Lite} is able to prodecue continuous sound in a series steps rather than make discrete sound step by step, which breaks the bound of the traditional music sequencer. Therefore, the pitch is changing continuously in \textit{CSketch Lite}. In figure \ref{fig: CSketch}, the yellow and blue line denote the trend of pitch changing. Take the top-left yellow line as an example, the pitch of the sound is continuously droping from G$\#$ to F. Even though, the pitch of the above music sequencer applications are still linear mapping.

Except for the linear mapping though the grids, some few Apps adopted the non-linear pitch mapping. For instance, \textit{Orbita} simulates the movemment of a small planet orbits around a central planet along an elliptical path. And in this case, different color of \textquotedblleft{planet}\textquotedblright represent different instruments, which produces sound while elliptical orbit. The pitch is changing continuously based on the distance between the small planet and the central planet(see figure \ref{fig: Orbita}).

\begin{figure}
  \includegraphics[width=12 cm]{images/CSketch_Lite.PNG}
  \centering
  \caption{CSketch Lite: grid-based, top-to-bottom and continuous pitch layout}
  \label{fig: CSketch}
\end{figure}

It is not unusal of mapping pitch to colour in music applications \citep{Reference14}. However, there was only one music sequencer found to represent pitch with different colors.  

\newpage
\bigskip
\begin{figure}
  \includegraphics[width=12 cm]{images/Orbita.PNG}
  \centering
  \caption{Orbita: elliptical orbita, non-linear and continuous pitch layout}
  \label{fig: Orbita}
\end{figure}
\bigskip

\section{Results}
\label{sec: result}
 % Experimental Setup

\chapter{Study 2: User Study}

Following the first study (See Chapter \ref{ch: chapter 3}), A laboratory study was conducted to evaluate user experience on different design patterns of music sequencers. Base on the previous work of evaluating music instruments, a questionnaire was designed to measure muscians experience (See Section \ref{subsec: questionnaire}).

\section{Method}
\subsection{Questionnaire}
\label{subsec: questionnaire}

Base on \citeauthor{Reference0}'s work, which developed a 80-item pool ordered by descending mean importance for questionnaire, 10 questions that scored the highest mark from 9 different categories were used in the user study (see Appendix \ref{app:Appendix A}).

\citeauthor{Reference0} indicated the following three criteria for musicians to perceive musical instruments:
\begin{flushleft}
  \qquad \qquad \quad\textbf{Experienced freedom and possibilities (EFP)}\\
  \qquad \qquad \quad\textbf{Perceived control and comfort (PCC)} \\
  \qquad \qquad \quad\textbf{Perceived stability, sound quality and aesthetics (PSSQA)}\\
\end{flushleft}
\textit{EFP} as the predominent facet, mainly targets at evaluating the musicianship and expressivity of music instruments. For example, questions like\textit{\textquotedblleft{The instrument allows me to express myself.}\textquotedblright} are used to decide whether the instruments can let muscians to express themselves; \textit{PCC} is used to assess the controbility of the music instruments. Questions such as \textit{\textquotedblleft{I can control the sound appropriately.}\textquotedblright} are setted to identify how well the musicians believed they can control the instruments; \textit{PSSQA} is the most unique facet which analyses the quality of the instruments from the material, the sound and the apperience perspectives. For instance, questions like\textit{\textquotedblleft The instrument pleases me sound-wise\textquotedblright} test the sound quality of the instrument. The above three interrelated facets construct the framework of MPX-Q questionnaire.

\begin{tabular}{ |p{1.2cm}|p{2.5cm}|p{9.2cm}|p{0.6cm}|}
 \multicolumn{4}{l}{Table} \\
 \hline
 Factor & Category  & Item  & $\mu$ \\
 \hline
 EFP & Creativity & The instrument allows me to be creative & 6.25\\
 & Enjoyment &  I have fun playing the instrument & 6.08\\
 & Expressiveness & The instrument allows me to express myself & 6.06\\
 \hline
 PCC & Conformance & The instrument responds well to my actions & 6.23\\
 & Control & I can control the sound appropriately & 6.04\\
 & Engagement & The instrument allows me to be engaged when I'm playing it & 5.98\\
 & Engagement & I feel the urge to play the instrument again & 5.79\\
 & Play Comfort & I can recognize that the instrument responds well to my playing & 5.85\\
 \hline
 PSSQA & Stability & I can rely on the instrument when playing it & 6.21\\
 & Sound Quality & The instrument pleases me sound-wise & 6.02\\
 \hline
\end{tabular}


Follow the framework of MPQ-Q questionnaire, 10 questions from 3 factors were implemented in our questionnaire. For each factor, only the items score the highest mean importance value in the certain category were picked. Under the EFP factor, we focused at the creativity, enjoyment and expressiveness of the music sequencer. The reason for this, it's because we want to figure out whether the design of the interface is encouraging musicians to explore new possibilities and inspiring musicians' creativity. As for the PCC, items associate with conformance, control and engagement are chose. The reason behind this is when musicians performing on instruments there are a lot of physical interaction between musicians and instruments, whether the musiciain feel conformance and engagement have impact on their overall satisfaction. For items under PSSQA, we only look at the stability and sound quality. Because the more stable of the music sequencer the more confident musicians can rely on it. Same with the sound quality, only the instrument that can satisfy the muscian is able to please the audience.
  
\subsection{Interview}

\section{Participants}



\section{Results}

\section{Discussion}

\section{Summary}
 % Experiment 1

\pagestyle{fancy}
\rhead{\thepage}
\lhead{Conclusion}
\chapter{Conclusion}
\label{ch: chapter 5}

This thesis has described studies that have been conducted to evaluating the touch-based music sequencer apps on iPad. In Chapter \ref{ch: chapter 2}, we illustrated the background of DMIs and highlighted the fact of apparent popularity of mobile devices. Also, we introduced the development of NIME community and current situation of evaluating the newborn musical interface. Then we carried out two consecutive studies to investigate factors that affect music sequencer interface's expressivity, control and aesthetics. In the first study, we examined 55 music sequencer downloaded from App Store. Results of the first study presented an interface taxonomy of current sequencer apps on iOS App Store (see Chapter \ref{ch: chapter 3}). And in the second study, followed by the interface taxonomy we concluded in the previous chapter, we perfomed an HCI user study with twenty musicians of the selectedapplications. We summarized the influence of different design appoachon on music sequencer interface (see Chapter \ref{ch: chapter 4}).

Through the evaluation of music sequencer on iPad, we identified the following contributions: \\
\qquad$\bullet$ An interface taxamony of music sequencer apps based on the mapping of interaction has been created, and which can potentially benefit future study on classifying touch-based musical interface on iPad or other similar mobile devices.\\
\qquad$\bullet$ Conducting an HCI user study and present an example questionnaire for analyzing musical interface on mobile devices.\\
\qquad$\bullet$ Providing design guidelines for future music sequencer development.

\section{Limitation and Future Work}

Associated with the contribution there are some limitations with the research. We only collected music sequencer applications from the iOS App Store. There are still more similar apps developed for Android OS have not been stuied.
The various music background of participants may affact their judgement on single senquencer apps. Last but not least, in the user study, we only analyzed the musicians opinions from a short impression.

Future work may expand the study object to Android applications, and further investigate the music sequencer interface from a long time study.



% \section{Future Work}
%
% Design guidline
% Develop our own sequencer
% Further investigate other music instrument application on the App Store by employing the user study.


\clearpage
 % Experiment 2

%\input{Chapters/Chapter6} % Results and Discussion

%\input{Chapters/Chapter7} % Conclusion

%% ----------------------------------------------------------------
% Now begin the Appendices, including them as separate files

\addtocontents{toc}{\vspace{2em}} % Add a gap in the Contents, for aesthetics

\appendix % Cue to tell LaTeX that the following 'chapters' are Appendices
\chapter{Independent Study Contract}

\vspace{-1.8cm}
\includegraphics[width=\textwidth]{images/IndependentStudyContract1}
\newpage

\includegraphics[width=\textwidth]{images/IndependentStudyContract2}



% \setlength{\voffset}{0cm}
% \setlength{\hoffset}{0cm}
%
% \includepdf[pages=-]{images/IndependentStudyContract.pdf}
%
% \setlength{\voffset}{-2.54cm}
% \setlength{\hoffset}{-2.54cm}
	% Appendix Title

\chapter{App Store Music Sequencer Applications}

\begin{tabular}{ |p{3cm}||p{3.5cm}|p{3.5cm}|p{3.5cm}||}
 \hline
 \multicolumn{4}{|c|}{App Store Music Sequencer Applications} \\
 \hline
 Application Name   & Description  & Seller & Link\\
 \hline
 Music Pad  & dj player remix electronic music beat & Xinggui Zhang & \url{<https://appsto.re/au/_Dkmeb.i>}\\

 Volotic & N/A & Scott Garner &
\url{ https://appsto.re/au/-WW64.i}\\

 Beatwave & N/A & collect3 &
 \url{https://appsto.re/au/UzERv.i}\\

 EGDR808 & Drum Machine free & Elliott Garage &
 \url{https://appsto.re/au/rPfXO.i}\\

 LoopStation & N/A & Rene Zuidhof &
 \url{https://appsto.re/au/UzMw7.i}\\

 Noise & N/A & ROLI Ltd &
 \url{https://appsto.re/au/Zzkr8.i}\\

 Music Strobe Starter & N/A & Arun Bab&
 \url{https://appsto.re/au/y4NFQ.i}\\

 Beatbox Looper & N/A & Pierre Guilluy&
 \url{https://appsto.re/au/Sfk6R.i}\\

 Dubstep Invasion & Music And Song Hit Maker & Jochen Heizmann&
 \url{https://appsto.re/au/Oane3.i}\\



 \hline
\end{tabular}
\fancyfoot[C]{\thepage}

\newpage
\pagestyle{fancy}
\fancyhead[L]{Appendix A}
\fancyfoot{}
\fancyfoot[C]{\thepage}

\begin{tabular}{ |p{3cm}||p{3.5cm}|p{3.5cm}|p{3.5cm}||}
 \hline
 \multicolumn{4}{|c|}{App Store Music Sequencer Applications(Continued)} \\
 \hline
 Application Name   & Description  & Seller & Link\\
 \hline

 Remix Pads & make groove beats record music app & Alexey Natarov&
 \url{https://appsto.re/au/R7_pdb.i}\\

 Music Touch & Make Mix Music DJ Beats & Qiao He&
 \url{https://appsto.re/au/D_ZTdb.i}\\

 Loop maker & Amazing music maker & Miguel Saldana &
 \url{https://appsto.re/au/MpDthb.i}\\

 Drum Pads Machine & Beat maker dj music studio & Alexey&
 \url{https://appsto.re/au/JZ9adb.i}\\

 Drum Pads Machine 2 & Beat maker dj music app & Alexey Natarov&
 \url{https://appsto.re/au/c5DZdb.i}\\

 MIxpads & Virtual dj pads sampler free app & Alexey Natarov &
 \url{https://appsto.re/au/CPj1eb.i}\\

 Loopacks & Music Maker Loop Machine DJ Beats & Hernan Arber &
 \url{https://appsto.re/au/oXKt1.i}\\

 Dubstep Dubpad 2 & Electronic Music Sampler & FAD Games LLC &
 \url{https://appsto.re/au/mCRXO.i}\\

 NOIZ & Make Epic Music & Studio Amplify &
 \url{https://appsto.re/au/KK9Uab.i}\\

 Blocs Wave & Make Record Music & Novation &
 \url{https://appsto.re/au/L0MTab.i}\\

 MIxpads 2 & Dubstep Trap drum pad sampler for DJ & Alexey Natarov &
 \url{https://appsto.re/au/oH_ffb.i}\\

 Polyphonic! & NA & Flip Studios LLC &
 \url{https://appsto.re/au/u_PhS.i}\\

 Steve Reich's Clapping Music & Improve Your Rhythm & Amphio Limited &
 \url{https://appsto.re/au/R-JA4.i}\\

 Music Pad  & remix electronic music beat & Xinggui Zhang &
 \url{https://appsto.re/au/_Dkmeb.i}\\

 Loop Community & NA & Loop Community &
 \url{https://appsto.re/au/VyLNN.i}\\

 LP-5 & Loop-based Music Sequencer & Markus Waldboth &
  \url{https://appsto.re/au/Z6EDN.i}\\

  \hline
 \end{tabular}

 \newpage
 \pagestyle{fancy}
 \fancyhead[L]{Appendix A}
 \fancyfoot{}
 \fancyfoot[C]{\thepage}

 \begin{tabular}{ |p{3cm}||p{3.5cm}|p{3.5cm}|p{3.5cm}||}
  \hline
  \multicolumn{4}{|c|}{App Store Music Sequencer Applications(Continued)} \\
  \hline
  Application Name   & Description  & Seller & Link\\
  \hline

 Dubstep Song Construction Kit &NA & Jochen Heizmann &
  \url{https://appsto.re/au/Knd0I.i}\\

 Dubstep Filth Factory & Sampler and Loop Machine & Ben Frost &
  \url{https://appsto.re/au/iHnUX.i}\\

 Monolith Loop & Relax Meditate Sleep Zen & Monolith Interactive Inc.&
  \url{https://appsto.re/au/vfGDy.i}\\

 Theremin Synth & Loop Record Download & Luke Phillips &
 \url{https://appsto.re/au/gJI2bb.i}\\

 Music Makr JAM & Create remix share your music! & JAM just add music GmbH &
 \url{https://appsto.re/au/EXEG0.i}\\

 Novation Launchpad & Make Remix Music & Novation &
 \url{https://appsto.re/au/QNk1I.i}\\

 Multi Track Song Recorder & NA & Derrick Walker &
 \url{https://appsto.re/au/Ygbsx.i}\\

 Triqtraq & Jam Sequencer music making on the go & Zaplin Music &
 \url{https://appsto.re/au/G8XhD.i}\\

 Trigger Box & NA & Justus Kandzi &
 \url{https://appsto.re/au/j4Hn1.i}\\

 Composer's Sketchpad Lite & NA & Alexei Baboulevitch &
 \url{https://appsto.re/au/nWJO_.i}\\

 Orbita for iOS & NA & Keijiro Takahashi&
 \url{https://appsto.re/au/kBIaN.i}\\

 S.A.M.M.I. & NA & Christopher Ayles &
 \url{https://appsto.re/au/YDMeY.i}\\

 ScratchVOX & NA & ScratchVOX &
 \url{https://appsto.re/au/e4aX0.i}\\

 Oro & Visual Music & Light the Music LLC &
   \url{https://appsto.re/au/d6px5.i}\\

 Poly & NA & James Milton &
   \url{https://appsto.re/au/LFspN.i}\\

 Mutone & NA & william LINDMEIER &
   \url{https://appsto.re/au/IkoJM.i}\\

  \hline
 \end{tabular}

\newpage
\pagestyle{fancy}
\fancyhead[L]{Appendix A}
\fancyfoot{}
\fancyfoot[C]{\thepage}

 \begin{tabular}{ |p{3cm}||p{3.5cm}|p{3.5cm}|p{3.5cm}||}
  \hline
  \multicolumn{4}{|c|}{App Store Music Sequencer Applications(Continued)} \\
  \hline
  Application Name   & Description  & Seller & Link\\
  \hline

  WR6000 & NA & WEJAAM &
  \url{https://appsto.re/au/pM3E3.i}\\

  SoundZen HD & NA & Tapbox LTD &
  \url{https://appsto.re/au/dHrZB.i}\\

  SoundGrid & NA & Vitaly Pronkin &
  \url{https://appsto.re/au/fSB3s.i}\\

  Visual Beat & Interactive Music Video & Max Moertl &
  \url{https://appsto.re/au/B-8l6.i}\\

  MINI-COMPOSER & NA & Masayuki Akamatsu &
  \url{https://appsto.re/au/Ar8Ez.i}\\

 Loopseque Lite & NA & Casual Underground &
 \url{https://appsto.re/au/BTm8x.i}\\

 Bass Drop & Deep House Electronic music sampler and synthesizer & Ben Frost &
 \url{https://appsto.re/au/k3rp0.i}\\

 Beat Boss & Electronic Dance Music Sampler & Ben Frost &
 \url{https://appsto.re/au/DWLyU.i}\\

 TonePad &NA& LoftLab &
 \url{https://appsto.re/au/nOx1s.i}\\

 Navichord Lite & intuitive chord sequencer & Denis Kutuzov &
 \url{https://appsto.re/au/kTci2.i}\\

 EasyBeats Drum Machine Free MPC & Hopefully Useful Software & Christian Inkster &
 \url{https://appsto.re/au/gJ1Ot.i}\\

 Fifth Degree & MIDI Sequencer & Bernie Maier &
 \url{https://appsto.re/au/qFZM1.i}\\

 Light Medley & NA & Tek Min Ewe &
 \url{https://appsto.re/au/FU06hb.i}\\

 Medly &  Music Maker & Medly Labs Inc &
 \url{https://appsto.re/au/CP1c4.i}\\

 \hline
\end{tabular}
	% Appendix Title

\chapter{Questionnaire}

% \pagestyle{fancy}
\fancyhead[L]{Appendix B}

% \begin{document}
\vspace{-1.8cm}
\includegraphics[width=\textwidth]{Questionnaire_1.png}

\includegraphics[width=\textwidth]{Questionnaire_2.png}
% \end{document}
 % Appendix Title

\chapter{Interview Questions}
\label{app: AppendixC}
\fancyhead[L]{Appendix C}

This document contains a list of questions that will be asked in the interview.

{\fontfamily{qcr}\selectfont
1. Can you tell me about your experience and training in music? How long have you been learning? \\

2. What kind of instrument you play most in your spare time, and what kind of instrument you instrument you prefer to play? \\

3. Among the 3 apps you just played, do you find one app attracted you most, or you think they are all very boring? \\

4. Have you heard about or used music sequencers before? \\

5. After playing the three apps, can identify any interface patterns in these music sequencer apps? \\

6. Did one particular interface most inspired your creativity? How? \\

7. Do you think the complexity of the interface has an effect on how enjoyable the app is to play? \\

8. Would you play any of these apps later, or will you tell your friends about them? \\
}
\clearpage
 % Appendix Title

\chapter{Questionnaire Data}
\label{app: Appendix D}

\includegraphics[width=\textwidth]{images/questionnaire_data1.png}
\includegraphics[width=\textwidth]{images/questionnaire_data2.png}
\includegraphics[width=\textwidth]{images/questionnaire_data3.png}

\clearpage
 % Appendix Title

\addtocontents{toc}{\vspace{2em}}  % Add a gap in the Contents, for aesthetics
\backmatter

%% ----------------------------------------------------------------
\label{Bibliography}
\lhead{\emph{Bibliography}}  % Change the left side page header to "Bibliography"
\bibliographystyle{unsrtnat}  % Use the "unsrtnat" BibTeX style for formatting the Bibliography
\bibliography{Bibliography}  % The references (bibliography) information are stored in the file named "Bibliography.bib"

\end{document}  % The End
%% ----------------------------------------------------------------
