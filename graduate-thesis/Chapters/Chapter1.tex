\chapter{Introduction}

With the rapid development of digital audio technolodge, people start to find out that computers are playing an increasingly important role in music. This new trend is providing unprecedented opportunities for people to create and manipulate sound. However, the flexibility of the digital technolodge is accompanied by confuse and uncertainty. As a result, thousands of new musicial forms built on computers have been created and released to the world. And it is natural to ask, what kinds of musical interfaces are taking better advantage of computers. To answer this question, researchers targeting at the better establised field of human-computer interaction. Correspondingly, a new community called NIME was born (see \ref{subsec: nime}).

\section{Background}
\label{sec: backgound}

\subsection{The development of NIME}
\label{subsec: nime}
% what is NIME and it's development
The New Interface for Musical Expression (NIME) is an international conference for musicians and researchers from all over the world to demonstrate their latest work on musical interface design \citep{Reference15}. It first started as a workshop at the Conference on Human Factors in Computing System (CHI) in 2001. After that, annually conferences have been held around the world. The hoster are research groups who devote themselves to interface design, human-computer interaction and computer music. The latest conference was held at Griffith University in Brisbane, Queensland, Australia in 2016.

In the last sixteen years,



\subsection{iPad: a new playground for musicians}
% we will first brefly introduce the ipad, and then illustrate the strength of ipad. also the reason why we major investigate on ipad

The iPad, a tablet computer with touchscreen display, has quickly occupied the market all around world since it's first release in 2010\citep{Reference2}. The emergence of iPad have provided a new platform for users to explore digital world \citep{Reference1}. After 7 generations, the usage of iPad has shifted from the extension of iPhone to a powerful pruductivity tool. In this shift, thousands of applications which was designed to utilise the larger touch screen has emerged. According to Daniel, there are over 1.5 million apps are currently hosted in the App Store and more than half of those apps are specifically designed for iPad\citep{lifewire}.


\section{Related Work}

\section{Research goals and motivation}

\section{Structure}

The research project was divided into two consecutive studies()
