\chapter{Study 2: User Study}

Following the first study (See Chapter \ref{ch: chapter 3}), A laboratory study was conducted to evaluate user experience on different design patterns of music sequencers. Base on the previous work of evaluating music instruments, a questionnaire was designed to measure muscians experience (See Section \ref{subsec: questionnaire}).

\section{Method}
\subsection{Questionnaire}
\label{subsec: questionnaire}

Base on \citeauthor{Reference0}'s work, which developed a 80-item pool ordered by descending mean importance for questionnaire, 10 questions that scored the highest mark from 9 different categories were used in the user study (see Appendix \ref{app:Appendix A}).

\citeauthor{Reference0} indicated the following three criteria for musicians to perceive musical instruments:
\begin{flushleft}
  \qquad \qquad \quad\textbf{Experienced freedom and possibilities (EFP)}\\
  \qquad \qquad \quad\textbf{Perceived control and comfort (PCC)} \\
  \qquad \qquad \quad\textbf{Perceived stability, sound quality and aesthetics (PSSQA)}\\
\end{flushleft}
\textit{EFP} as the predominent facet, mainly targets at evaluating the musicianship and expressivity of music instruments. For example, questions like\textit{\textquotedblleft{The instrument allows me to express myself.}\textquotedblright} are used to decide whether the instruments can let muscians to express themselves; \textit{PCC} is used to assess the controbility of the music instruments. Questions such as \textit{\textquotedblleft{I can control the sound appropriately.}\textquotedblright} are setted to identify how well the musicians believed they can control the instruments; \textit{PSSQA} is the most unique facet which analyses the quality of the instruments from the material, the sound and the apperience perspectives. For instance, questions like\textit{\textquotedblleft The instrument pleases me sound-wise\textquotedblright} test the sound quality of the instrument. The above three interrelated facets construct the framework of MPX-Q questionnaire.

\begin{tabular}{ |p{1.2cm}|p{2.5cm}|p{9.2cm}|p{0.6cm}|}
 \multicolumn{4}{l}{Table} \\
 \hline
 Factor & Category  & Item  & $\mu$ \\
 \hline
 EFP & Creativity & The instrument allows me to be creative & 6.25\\
 & Enjoyment &  I have fun playing the instrument & 6.08\\
 & Expressiveness & The instrument allows me to express myself & 6.06\\
 \hline
 PCC & Conformance & The instrument responds well to my actions & 6.23\\
 & Control & I can control the sound appropriately & 6.04\\
 & Engagement & The instrument allows me to be engaged when I'm playing it & 5.98\\
 & Engagement & I feel the urge to play the instrument again & 5.79\\
 & Play Comfort & I can recognize that the instrument responds well to my playing & 5.85\\
 \hline
 PSSQA & Stability & I can rely on the instrument when playing it & 6.21\\
 & Sound Quality & The instrument pleases me sound-wise & 6.02\\
 \hline
\end{tabular}


Follow the framework of MPQ-Q questionnaire, 10 questions from 3 factors were implemented in our questionnaire. For each factor, only the items score the highest mean importance value in the certain category were picked. Under the EFP factor, we focused at the creativity, enjoyment and expressiveness of the music sequencer. The reason for this, it's because we want to figure out whether the design of the interface is encouraging musicians to explore new possibilities and inspiring musicians' creativity. As for the PCC, items associate with conformance, control and engagement are chose. The reason behind this is when musicians performing on instruments there are a lot of physical interaction between musicians and instruments, whether the musiciain feel conformance and engagement have impact on their overall satisfaction. For items under PSSQA, we only look at the stability and sound quality. Because the more stable of the music sequencer the more confident musicians can rely on it. Same with the sound quality, only the instrument that can satisfy the muscian is able to please the audience.
  
\subsection{Interview}

\section{Participants}



\section{Results}

\section{Discussion}

\section{Summary}
