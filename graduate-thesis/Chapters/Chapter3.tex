\chapter{Study 1: Classification of music sequencer}
\label{ch: chapter 3}

A big scale study was conducted to create an interface taxonomy of current music sequencer apps on the iOS App Store. In total, 55 music sequencer applications on App Store have been examined (see Appendix \ref{app:Appendix A}). Several search criteria are implemented to locate music sequencer on the App Store (see Section\ref{subsec: search criteria}). After analyzing those music sequencer apps, we proposed classification criteria based on the design of the user interface (see Section \ref{sec: classify criteria}). The 55 music sequencer applications were classify into 3 major groups according to the classification criteria (see Section\ref{sec: result}).

\section{Method}
\label{sec:method}

In total, 71 musical iOS applications associated with music sequencer had been downloaded from App Store. After examined and discussed with my supervisor Ben Swift, 16 applications were removed from the study list either because the application can hardly be classified as music sequencer or because the application was not designed for iPad. The rest 55 music sequencer applications were studied in detailed.

\subsection{Search Criteria}
\label{subsec: search criteria}

Base on \citeauthor{Reference13}'s study which created a whitelisted words for music sequencer, keywords such as \textit{Sequence, Sequencer, Groovebox, Beatbox, Step, MIDI, Pattern, Tempo, BPM, Machine} were used to search on the App Store. Before each application been downloaded, it's description had been briefly overviewed to make sure it was designed for music purpose. Also, in the searching criteria, \textquotedblleft{iPad only}\textquotedblright was chose and results were sorted under the relevance of keywords.

\subsection{Classification criteria}
\label{sec: classify criteria}
The different approaches of interacting with the applications were used to classify the user interface of the music sequencer applications into several categories. The mappings of the sequencer were broke down into 4 operations,
which were \textit{changing pitch, triggering sound, timing and changing timber}.

\textbf{Changing Pitch.} Becasue the way most traditional instruments' pitch were changed discretely(for example, piano, guitar and violin) most of the sequencer application  


\section{Results}
\label{sec: result}
