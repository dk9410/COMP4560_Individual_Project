\pagestyle{fancy}
\rhead{\thepage}
\lhead{Study 1: Classification of music sequencer}
\chapter{Study 1: Classification of music sequencer}
\label{ch: chapter 3}

To achieve the goal of creating an interface taxonomy of current music sequencer apps onthe iOS App Store, a survey was conducted to analyze the current situation of sequencer apps at App Store. In total, 55 music sequencer applications have been examined (see Appendix \ref{app:Appendix A}). Several search criteria are implemented to locate music sequencer on the App Store (see Section\ref{subsec: search criteria}). After analyzing those music sequencer apps, we proposed classification criteria based on the design of the user interface (see Section \ref{sec: classify criteria}). The 55 music sequencer applications were classify into 3 major groups according to the classification criteria (see Section\ref{sec: result}).

\section{Method}
\label{sec:method}

\subsection{Search Criteria}
\label{subsec: search criteria}

\citeauthor{Reference13}\textquotesingle s created a list of whitelisted words for music sequencer apps, keywords such asrt \textit{Sequence, Sequencer, Groovebox, Beatbox, Step, MIDI, Pattern, Tempo, BPM, Machine} were used to search on the App Store. Before each application was downloaded, it\textquotesingle s description had been briefly overviewed to make sure it was designed for the purpose of making music.
Also, when searching in the App Store, filter called \textquotedblleft{iPad Only}\textquotedblright was chose to show the result only assocaite with iPad. And the search results were sorted under the relevance of keywords.

In total, 71 musical iOS applications associated with music sequencer had been downloaded from App Store. After examining them in details, 16 applications were removed from the study list either because the application could hardly be classified as music sequencer or caused by the application was not designed for iPad. The rest 55 music sequencer applications were studied in detailed (see Appendix \ref{app:Appendix A}).

\subsection{Classification criteria}
\label{sec: classify criteria}
The different approaches of interacting with the applications were used to classify the user interface of the music sequencer applications into several categories. The mappings of the sequencer were broke down into 4 operations,
which were \textit{changing pitch, triggering sound, timing and changing timbre}.

\textbf{Changing Pitch.} Most traditional instruments' pitch were changed discretely, for example, piano, guitar and violin. The majority of musical application including sequencer follow this trend. Besides, pitch is dominated by grid-like, buttom-to-top mapping in music sequencer hardware. Therefore, grid-based, buttome-to-top and discrete pitches layout is widely adopted.

\bigskip
\begin{figure}[h]
  \includegraphics[width=12 cm]{images/Beatwave.PNG}
  \centering
  \caption{Beatwave: grid-based, buttome-to-top and discrete pitch layout}
  \label{fig: Beatwave}
\end{figure}
\bigskip

 Figure \ref{fig: Beatwave} is a good example of this classic interface, in which the interface is divided into 16x16 grids. The time, which is separated into 16 steps, only moves one step at a time from left to right. The blue vertical line works as a reminder of current time, and also indicates what is coming next(in the next step). The white square, on the other hand, represents the sound of a certain instrument. In this case, it represents an electrical sound called \textit{FUTURE}. The column in each step is divided into 16 scales and which are the pitches of the instrument. The white squares located in the top of the grids are high pitch sound of the instrument, on the contrary, the pitch of the sound from the bottom is relatively low.

 \bigskip
 \begin{figure}[h]
   \includegraphics[width=12 cm]{images/SoundZen.PNG}
   \centering
   \caption{SoundZen: grid-based, right-to-left and discrete pitch layout}
   \label{fig: SoundZen}
 \end{figure}
 \bigskip

 However, not all the grid-based sequencer applications increase pitch from bottom to top. There is a small portion of sequencer increase pitch from left to right. For instance, SoundZenHD used a left-to-right pitch mapping (see figure: \ref{fig: SoundZen}).

In addition to the discrete pitch mapping, there are attempts to implement the continuous pitch. \textit{CSketch Lite} followed the classic grid-based layout, but it implements continuous sequencing (see Figure \ref{fig: CSketch}). By implementing the continuous sequenceing, \textit{CSketch Lite} is able to prodecue continuous sound in a series steps rather than make discrete sound step by step, which breaks the bound of the traditional music sequencer. Therefore, the pitch is changing continuously in \textit{CSketch Lite}. In figure \ref{fig: CSketch}, the yellow and blue line denote the trend of pitch changing. Take the top-left yellow line as an example, the pitch of the sound is continuously droping from G$\#$ to F. Even though, the pitch of the above music sequencer applications are still linear mapping.

Except for the linear mapping though the grids, some few Apps adopted the non-linear pitch mapping. For instance, \textit{Orbita} simulates the movemment of a small planet orbits around a central planet along an elliptical path. And in this case, different color of \textquotedblleft{planet}\textquotedblright represent different instruments, which produces sound while elliptical orbit. The pitch is changing continuously based on the distance between the small planet and the central planet(see figure \ref{fig: Orbita}).

\bigskip
\begin{figure}[h]
  \includegraphics[width=12 cm]{images/CSketch_Lite.PNG}
  \centering
  \caption{CSketch Lite: grid-based, top-to-bottom and continuous pitch layout}
  \label{fig: CSketch}
\end{figure}
\bigskip

It is not unusal of mapping pitch to colour in music applications \citep{Reference14}. However, there was only one music sequencer found to represent pitch with different colors (see Figure \ref{fig: Volotic}). In \textit{Volotic}, there is an emiter which continuously emits red little dot sequently. The red color, in this case, means note C or \textbf{Do} which is the first note of the fixed-Do solfège scale. Once the red little dot passed though a tone assigner, it's tone changed relatively and so as it's color. In Figure \ref{fig: Volotic}, the green symbol is a tone assigner called TUNNING, and the number in the middle denotes what note it is going to assign. There are seven different TUNNINGs which together consist the key of C(or C major).

However, this color-based mapping is not intuitive. It takes significant effort to link different keys to colors. Besides, in this case, the two colors between pitch E (\textbf{Mi}, the third note of the C major scale) and pitch F (\textbf{Fa}, the fourth note of the C major scale) is very difficult to distinguish. This unintuitive mapping could be the reason why the color-based pitch is not widely implemented, and we will look into the details in the next chapter.

\bigskip
\begin{figure}[h]
  \includegraphics[width=12 cm]{images/Orbita.PNG}
  \centering
  \caption{Orbita: elliptical orbita, non-linear and continuous pitch layout}
  \label{fig: Orbita}
\end{figure}
\bigskip

\textbf{Triggering and Timing.} In \citeauthor{Reference14}'s study, the mechanics of how users inteacted with applications and the methods of how time was represented were studied seperately. However, in most music sequencer applications, time is used to triiger sounds. Therefore, triggering and timing were analyzed together in our study.

Unsuperisingly, given the fact that toggles are primary used on sequencer hardware, virtual toggles are the most commonl method for users to control sequencer applications to start producing sounds. In Figure \ref{fig: Beatwave}, \ref{fig: SoundZen} and \ref{fig: CSketch} there are virtual toggles acting as main switch to control the play/stop operation. After the main switch turned on, time is uesd to determaine the triggering sequence of a series of notes or several pieces of sounds. Likewise, timing in the majority of sequencer applications follow the convention of sequencer hardware, which time move from left to right. Some very few applications don't have an explicity display of time, such as \textit{Orbita} and \textit{Volotic} (see Figure \ref{fig: Orbita}, \ref{fig: Volotic}).

\bigskip
\begin{figure}[h]
  \includegraphics[width=12 cm]{images/Volotic.PNG}
  \centering
  \caption{Volotic: game-like, linear and color-based pitch layout}
  \label{fig: Volotic}
\end{figure}
\bigskip

\textbf{Timber and Volumn.} The majority of music sequencer applications use toggles to change timbers and volumn. Normally, there are several preset timbers and users can shifted between different timbers by selecting one of the preset timbers. Only a very small number of applications use additional control over timber. \textit{Volotic} uses the the symbol of different instruments to represent the unique timbers (see Figure \ref{fig: Volotic}. Essentially, it is still a toggle but in a twisted form.

The reason why mapping of volumn is combined with timber is the majority of music sequencer use the same mapping which is a slider or toggle. Other volumn controls are very rare. \textit{Orbita} is the only example of using the distance between the satellite and the planet to control the volumn. The volumn is turn up when satellite get close to the central planet. Conversely, the volumn goes down when two planets move apart.



\section{Results}

According to the classification criteria illustrated on section \ref{sec: classify criteria}, we divided the music sequencers' interfaces into the following three categories. Novel Interfaces, Multi-track Interface and Traditional Interface.

\textbf{Novel Interface.} In total, thirteen applications
were classified into this category. Novel interface was probably the most intuitive classification, because the apps under this group use entirely different mapping of time and timber. For example, \textit{Volotic} deployed a game-like sequencer interface (see Figure \ref{fig: Volotic}), and \textit{Orbit} imitated the celestial motion in the interface design. The list of apps under novel category were given in figure \ref{fig: Novel} below.

\bigskip
\begin{figure}[h]
  \includegraphics[width=0.5\textwidth]{images/Novel.png}
  \centering
  \caption{Apps under the Novel category}
  \label{fig: Novel}
\end{figure}
\bigskip

\textbf{Traditional Interface.} There were twenty-five apps belong to this category. The majority of music sequencer apps in this group followed the layout of sequencer hardware, and which was grid-based paradigm. The applications with traditional interface were listed in figure \ref{fig: Traditional}.

\bigskip
\begin{figure}[h]
  \includegraphics[width=0.50\textwidth]{images/Traditional1.png}
  \includegraphics[width=0.50\textwidth]{images/Traditional2.png}
  \caption{Apps under the Traditional category}
  \label{fig: Traditional}
\end{figure}
\bigskip

\textbf{Multi-track Interface} There were eighteen applications grouped under multi-track interface category. Essentially, multi-track interface still followed grid-based. But an extra management system was developed to edit different layers of sounds, in which the trigger of different sound can happen simultaneously. For instance, \textit{Beatwave} the most representative application of this kind, can play four different traks at the same time. The applications in this category were listed below (See Figure \ref{fig: Multi}).

\bigskip
\begin{figure}[h]
  \includegraphics[width=0.505\textwidth]{images/Multi-track2.png}
  \includegraphics[width=0.495\textwidth]{images/Multi-track1.png}
  \caption{Apps under the Multi-track category}
  \label{fig: Multi}
\end{figure}
\bigskip
\clearpage
