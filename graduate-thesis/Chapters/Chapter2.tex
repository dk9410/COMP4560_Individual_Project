\chapter{Literature review}

\section{Mobile Music}

With the increasing popularity of mobile device such as smart phone and tablet, a new research field called Mobile Music emerged \citep{Reference4}. According to the definition by \citeauthor{Reference6}, \textit{Mobile Music} wich employing portable technolodge does not only include the scope of playing music, but also involve music composing, synthesizing and sharing\citep{Reference6}.

In the last 15 years, there is a growing number of researchers start concerning the development of applications in mobile devices. This new trend was first highlighted by \citeauthor{Reference12} after analysing 98 NIME procedding papers related to mobile music during the period from 2002 to 2012\citep{Reference12}.

The expanding capabilities of mobile devices inspired researchers to exploit the new features.The wireless network ability of mobile device is the first area attract researchers' attention. TunA is the first practice of building connection among PDA users though wireless network\citep{Reference7}. By accessing the playlists of nearby users, TunA help users in same network to exchange their music. \citeauthor{Reference5} extended \citeauthor{Reference7}'s work from music sharing towards collaborative musical creation \citep{Reference5}. \citeauthor{Reference5} propsed a system which exploits ad-hoc wireless networks to allow a community of people using their PDA to work on the same piece of music \citep{Reference5}. Some research started from a different approach by investigating the possibility of utilizing the touch screen on the mobile devices. Geiger designed a paradigm for using touch screen on mobile device like iPaq \citep{Reference9, Reference10}.
MoGMI, which stand for Mobile Gesture Music Instrument, is a research project focused on using the accelorometer inside the mobile phone to perform music. Through examing three different axis mapping models, \citeauthor{Reference11} explored how to turn mobile phone into a standard instrument. Smule Ocarina is the most successful mobile musical artifact, which take advantanges of the global popularity of iPhone \citep{Reference8.1}. It leveraged the microphone to take input from breath, and combined with command from the multitouch screen to mimic the physical interaction of ocarina. Besides, Smule Ocarina also utilizing the GPS module to connect users all around the world and create a new social experience \citep{Reference8}.


\section{Musical Interaction Patterns}
% in this section we are going to introduce several major interaction patternn in mobile devices bath on \cite{Reference4}

\section{Evaluation of musical instruments}
\subsection{From performer's pespective}
\subsection{rom audience's pespective}
