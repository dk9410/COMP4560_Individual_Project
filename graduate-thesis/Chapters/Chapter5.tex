\pagestyle{fancy}
\rhead{\thepage}
\lhead{Conclusion}
\chapter{Conclusion}
\label{ch: chapter 5}
\section{Summary}

With the repid development of technoldge, mobile devices have become the new playground for musicians to express themselves. With a variety of sensors, as well as the exponentional growth in the processing power, iPad offer an attracitve platform for music performing. Thousands of music applications have been developed for the iPad. Music sequencer applications, as one of the major category of music making applications, have seen a lot of derivation and innovation. But the question is among those novel music sequencer interface which one supports musicians performance and stimulates their creativities. This thesis is trying to answer the above question from the perspective of musicians. In the first study, we analyzed 55 music sequencer applications from App Store and created an interface taxonomy. The music sequencer applications were divided into three groups according to the mapping of pitch, trigger and timber. In the second consecutive study, three most representative applications from each group were selected and tested by musicians to evaluate the pros and cons in the different design approaches. By employing the MPX-Q questionnaire, we quantitatively analyzed the strength and weakness of each sequencer applications. Follow by, a qualitative study was conducted to further investigate the reason behind those design.

\section{Limitation}

Only collect apps from iOS App Store, there are more from android market.

It's hard to say the selected apps are the best representative from their own class.



\section{Future Work}

Design guidline
Develop our own sequencer
Further investigate other music instrument application on the App Store by employing the user study.


\clearpage
