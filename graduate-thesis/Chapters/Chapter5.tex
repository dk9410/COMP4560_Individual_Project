\pagestyle{fancy}
\rhead{\thepage}
\lhead{Conclusion}
\chapter{Conclusion}
\label{ch: chapter 5}

This thesis has described studies that have been conducted to evaluating the touch-based music sequencer apps on iPad. In Chapter \ref{ch: chapter 2}, we illustrated the background of DMIs and highlighted the fact of apparent popularity of mobile devices. Also, we introduced the development of NIME community and current situation of evaluating the newborn musical interface. Then we carried out two consecutive studies to investigate factors that affect music sequencer interface's expressivity, control and aesthetics. In the first study, we examined 55 music sequencer downloaded from App Store. Results of the first study presented an interface taxonomy of current sequencer apps on iOS App Store (see Chapter \ref{ch: chapter 3}). And in the second study, followed by the interface taxonomy we concluded in the previous chapter, we perfomed an HCI user study with twenty musicians of the selectedapplications. We summarized the influence of different design appoachon on music sequencer interface (see Chapter \ref{ch: chapter 4}).

Through the evaluation of music sequencer on iPad, we identified the following contributions: \\
\qquad$\bullet$ An interface taxamony of music sequencer apps based on the mapping of interaction has been created, and which can potentially benefit future study on classifying touch-based musical interface on iPad or other similar mobile devices.\\
\qquad$\bullet$ Conducting an HCI user study and present an example questionnaire for analyzing musical interface on mobile devices.\\
\qquad$\bullet$ Providing design guidelines for future music sequencer development.

\section{Limitation and Future Work}

Associated with the contribution there are some limitations with the research. We only collected music sequencer applications from the iOS App Store. There are still more similar apps developed for Android OS have not been stuied.
The various music background of participants may affact their judgement on single senquencer apps. Last but not least, in the user study, we only analyzed the musicians opinions from a short impression.

Future work may expand the study object to Android applications, and further investigate the music sequencer interface from a long time study.



% \section{Future Work}
%
% Design guidline
% Develop our own sequencer
% Further investigate other music instrument application on the App Store by employing the user study.


\clearpage
